\section{Discussion}
\label{sec:discussion}
Starting with the estimation for generated heat, the results show that approximately 33\% of the nominal engine power is converted to heat. These results can not be taken at face value due to the difficulty of finding reliable internal resistance values for the specified pouch cell and the simplified applied model for heat generation, but it is good estimate for use as an initial approach.

This preliminary study showed that a passive battery cooling system based only on heat conduction through the wing skin is clearly feasible for a race airplane. This is very promising because this solution has virtually zero weight and drag penalty. It also has the structural advantage of keeping the batteries inside the wings, since they make up a considerable part of the weight of the airplane. This approach, however has disadvantages for maintenance, as it can become hard to access the batteries in this configuration. Solutions for this problem include installing removable panels on the lower surface of the wings, but this comes with added structural weight, and performance tends to be favored over ease of maintenance for racer aircraft.

From \cref{fig:surfacetemp} it is possible to conclude that low heat transfer coefficient values have a major impact on battery temperature. This means that maintaining appropriate battery temperature during takeoff might be a problem. However, this problem is mitigated due to the fact that the takeoff period is relatively short and during this period there are still a thermal transient. Therefore, the battery's own heat capacity prevents overheating. This justifies the consideration of only steady state heat transfer in this study, but for cold days this transient may be the active design constraint.

Another problem that might arise is taxiing or parking for prolonged periods of time under the sun, but, since this is a racer aircraft, this time can be made very short.

In general, for  racer aircraft, the flight envelope can be reduced to favor performance. For example, the flight could be performed in a cold day in order to achieve reduced battery temperatures. The operation could be aggressively restricted to ISA or even ISA$-10$ temperatures, and a suitable location chosen for flight test. The batteries could also be cooled prior to takeoff, using thermal blankets, mitigating the heating during this flight phase.

\Cref{fig:materials} shows that for this model, heat conductivity of the wing skin has a negligible effect on battery temperature, since the three curves are overlapping. This is probably due to the wing skin being very thin, but further analysis should be made in order to validate this claim.

In contrast to the wing skin, battery thickness can have a major influence on surface temperature as shown in \cref{fig:bathick}. 

The results for the air cooling approach are clearly absurd. The values obtained for mass flow and drag are more representative of a jet engine than a cooling device. Clearly the packs were made too big. If they are made smaller, the surface available for heat transfer would increase and thus the mass flow and drag will fall to reasonable values. Oxys Energy packs are already designed with cooling paths that alleviate this problem, but this was not included in the model, thus compromising the analysis.

It is important to mention that, in this study, the main concern was keeping the battery temperature in the manufactured specified range (which is between 5 and 30 degrees). This is in line with the climate in Brazil, where the airplane will be manufactured and probably flown. The same study could have been made for cold days, for example in subzero conditions, thus seeking ways to insulate or even heat the battery.

For future work, the battery heat generation model should be made more precise, and the transient takeoff phase should be considered. The compromise between expanding the flight temperature envelope or favoring performance should be better explored. Finally, the air cooling system should be better designed in order to yield representative results.