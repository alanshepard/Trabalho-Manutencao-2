\section{Conclusion}
\label{sec:conclusion}

In this paper a preliminary design approach was discussed for the battery cooling system of a racer airplane. It was shown that it is feasible to design a system based on heat conduction through the lower surface of the wings. It was shown that restricting the flight envelope (i.e.\ flying only on cold days) can improve performance by allowing the use this kind of cooling system. Further study should be made on the topic in order to validate these preliminary results and serve as subside for the final airplane design.

%A thermal management system designed for a racer airplane regulates the battery pack temperatures to the desired operating range. However the design aims at a very specific goal which is to  be the fastest during a very short amount of time. %An air cooling system can remove a significant amount of heat, nonetheless its impact on the drag is huge as modeled. A better distribution of the air between cells could minimize the drag. The battery surface temperature have also a strong impact on the heat transfer coefficient, some batteries can handle higher temperatures for a short period of time and thus determine whether the mission can be accomplished. 
